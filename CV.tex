%%%%%%%%%%%%%%%%%%%%%%%%%%%%%%%%%%%%%%%%%
% Important note:
% This template requires the resume.cls file to be in the same directory as the
% .tex file. The resume.cls file provides the resume style used for structuring the
% document.
%
%%%%%%%%%%%%%%%%%%%%%%%%%%%%%%%%%%%%%%%%%

%----------------------------------------------------------------------------------------
%	PACKAGES AND OTHER DOCUMENT CONFIGURATIONS
%----------------------------------------------------------------------------------------

\documentclass{resume} % Use the custom resume.cls style

\usepackage [autostyle, english = american]{csquotes}
\MakeOuterQuote{"}

\usepackage{xcolor}
\pagecolor{white}

\usepackage{xstring}
\usepackage{bibentry}
\usepackage[left=0.75in,top=0.6in,right=0.75in,bottom=0.6in]{geometry} % Document margins
\newcommand{\tab}[1]{\hspace{.2667\textwidth}\rlap{#1}}
\newcommand{\itab}[1]{\hspace{0em}\rlap{#1}}
\name{Daniel A. Snellings} % Your name
\address{710D South LaSalle Street, Durham, NC 27705} % Your address
\address{271 CARL Building, Duke University} % Your secondary addess (optional)
\address{301-885-6216 \\ daniel.snellings@duke.edu} % Your phone number and email

\def\FormatName#1{%Bolds any instance of Snellings in the bibliography
  \IfSubStr{#1}{Snellings}{\textbf{#1}}{#1}%
}

\begin{document}
\nobibliography{refs.bib}

\bibliographystyle{CV}


%----------------------------------------------------------------------------------------
%	EDUCATION
%----------------------------------------------------------------------------------------

\begin{rSection}{Education}
%--copy and paste this region  if you need more--
{\bf Ph.D. Molecular Genetics and Microbiology } \hfill {2017 - Present} 
\\ Program in Cell and Molecular Biology
\\ Duke University
\\ \\
%--copy and paste this region  if you need more--.
{\bf B.S. Biochemistry and Molecular Biology} \hfill {2013 - 2017} 
\\ Pennsylvania State University
\\
\end{rSection}


%----------------------------------------------------------------------------------------
%	RESEARCH
%----------------------------------------------------------------------------------------
\begin{rSection}{Research}
%--copy and paste this region  if you need more--
{\bf The Role of Somatic Mutations in Vascular Malformations} \hfill {2017 - Present} \\
\textit{Douglas A Marchuk, Duke University} \\
My work in the Marchuk Lab focuses on the genetic changes that lead to
hereditary and sporadic neurovascular malformations. Specifically, I have shown
that vascular malformations in Hereditary Hemorrhagic Telangiectasia follow a Knudsonian
two-hit mechanism; and that cerebral cavernous malformations accumulate multiple synergistic somatic
mutations which contribute to pathogenesis.  \\\\
%--copy and paste this region  if you need more--
{\bf Environmental Factors Influencing Bumblebee Pigmentation} \hfill {Academic Year 2014 - 2017} \\
\textit{Heather M Hines, Pennsylvania State University} \\
In the Hines Lab I studied the mechanism of pigment biosynthesis and deposition in developing bumblebees.
I also investigated the impact of foraging success and nutrient diversity on the pigment intensity of adult bees
for potential use in the field as a bioindicator of nutritional fitness. \\\\
%--copy and paste this region  if you need more--
{\bf The Mechanism of Cement Production in Barnacles} \hfill {Summers 2015 - 2016} \\
\textit{Christopher M Spillmann, Naval Research Laboratory} \\
At the Naval Research Lab I worked with a group focused on understanding the biological mechanism of
barnacle cement production and deposition with the ultimate goal of developing a hull coating which 
would prevent barnacle biofouling of naval vessels. Towards this end, I studied a previously
undescribed tissue and helped characterize its role in barnacle development. \\
\end{rSection}


%--------------------------------------------------------------------------------
%    PUBLICATIONS
%-----------------------------------------------------------------------------------------------
\begin{rSection}{Publications}
{* Authors contributed equally}
\\ \\ {\bf 2021} \\
\bibentry{Ren:2021aa}
\\ \\{\bf 2019} \\
\bibentry{Snellings:2019aa}
\bibentry{Koskimaki:2019aa}
\\ \\ {\bf 2018}\\
\bibentry{Detter:2018aa}
\bibentry{Wang:2018aa}
\\
\end{rSection}


%--------------------------------------------------------------------------------
%    SOFTWARE
%-----------------------------------------------------------------------------------------------
\begin{rSection}{Software}
{\bf gonomics} {(github.com/vertgenlab/gonomics)} \hfill{Role: Developer}
\\ {A collection of genomics software tools written in Go (golang).}
\\ {My work in gonomics focuses on developing a somatic variant caller that operates on sequencing data 
aligned to traditional linear references as well as data aligned to graph references as have been implemented
in gonomics.}

{\bf weaver} {(github.com/ddsnellings/weaver)} \hfill{Role: Creator \& Developer}
\\ {An open source toolkit for analyzing sequencing data generated by the Tapestri platform.}
\\ {Currently available analysis pipelines for the Tapestri platform are proprietary and are only available through an 
AWS instance provided by Mission Bio. I have written several functions to analyze data I have generated with this
platform. Though still very new, weaver is an outlet for me to formalize these functions for future use, and hopefully 
be a useful resource for the community.}

\end{rSection}

%--------------------------------------------------------------------------------
%    FUNDING
%-----------------------------------------------------------------------------------------------
\begin{rSection}{Funding}
{\bf F31 NIH/NHLBI  }{(1F31HL152738-01)} \qquad{Role: PI} \hfill {April 2020 - March 2023}\\
Investigating the Role of Somatic Mutations in Arteriovenous Malformations\\
\end{rSection}


%--------------------------------------------------------------------------------
%    PRESENTATIONS
%-----------------------------------------------------------------------------------------------
\begin{rSection}{Selected Presentations}
%--copy and paste this region  if you need more--
{\bf Invited Mission Bio Tapestri Webinar} \hfill {February  2021}\\
Talk: "Multiple Somatic Mutations in a Single Clonal Population Drive CCM Pathogenesis"\\\\
%--copy and paste this region  if you need more--
{\bf American Society of Human Genetics 2020 Annual Meeting} \hfill {October 2020}\\
Poser 1720: "A Novel Mutation in \textit{GNAQ} Identified in Sturge-Weber Syndrome"\\\\
%--copy and paste this region  if you need more--
{\bf American Society of Human Genetics 2019 Annual Meeting} \hfill {October 2019}\\
Flash Talk: "A Genetic Two-Hit Mechanism Drives Vascular Malformation in HHT"\\\\
%--copy and paste this region  if you need more--
{\bf American Society of Human Genetics 2019 Annual Meeting} \hfill {October 2019}\\
Poster 1238/F: "A Genetic Two-Hit Mechanism Drives Vascular Malformation in HHT"\\\\
%--copy and paste this region  if you need more--
{\bf 13th HHT International Scientific Conference} \hfill {June 2019}\\
Talk: "HHT Telangiectases Contain Biallelic Mutations in \textit{ENG} or \textit{ACVRL1}"\\

\end{rSection}


%--------------------------------------------------------------------------------
%    OUTREACH
%-----------------------------------------------------------------------------------------------
\begin{rSection}{Outreach}
%--copy and paste this region  if you need more--
{\bf Undergraduate Career Development Panel} \hfill {October 2019}\\
Served as a panelist detailing my path to graduate school and discussed career options with 1st year undergraduates.\\\\
%--copy and paste this region  if you need more--
{\bf The Great Insect Fair} \hfill {May 2016}\\
Displayed samples and taught children about the importance of bumblebee coloration and the presence of color mimics in the wild.\\

\end{rSection}


%--------------------------------------------------------------------------------
%    MENTORSHIP
%-----------------------------------------------------------------------------------------------
\begin{rSection}{Mentorship}
%--copy and paste this region  if you need more--
{\bf Jeff Reitano, Rotation Student} \hfill {2021}\\
%--copy and paste this region  if you need more--
{\bf Daichi Shonai, Rotation Student} \hfill {2021}\\
%--copy and paste this region  if you need more--
{\bf Makenzie Beaman, Rotation Student} \hfill {2020}\\
%--copy and paste this region  if you need more--
{\bf Taylor Anglen, Rotation Student} \hfill {2020}\\
%--copy and paste this region  if you need more--
{\bf Nicole Kastelic, Undergraduate Researcher} \hfill {2019 - 2020}\\
%--copy and paste this region  if you need more--
{\bf Makala Moore, Rotation Student} \hfill {2019}\\
%--copy and paste this region  if you need more--
{\bf Layne Clements, Undergraduate Summer Student} \hfill {2018}\\
%--copy and paste this region  if you need more--

\end{rSection}


%----------------------------------------------------------------------------------------
%	PROFESSIONAL MEMBERSHIPS
%----------------------------------------------------------------------------------------
\begin{rSection}{Professional Memberships}
{\bf American Society of Human Genetics} (ASHG)  \hfill {2019 - Present} 
\\ \\
{\bf American Heart Association} (AHA) \hfill {2019 - Present} 
\\ \\
{\bf American Association for the Advancement of Science} (AAAS) \hfill {2019 - Present} 
\\
\end{rSection}


%----------------------------------------------------------------------------------------
%	AWARDS
%----------------------------------------------------------------------------------------
\begin{rSection}{Honors and Awards}
{\bf Reviewers Choice Abstract} ASHG 2019 Annual Meeting \hfill {October 2019} \\ \\
{\bf Best Scientific Oral Presentation} 13th HHT International Scientific Conference \hfill {June 2019} \\ \\
{\bf Molecular Genetics and Microbiology Travel Award} Duke University \hfill {April 2019} \\ \\
{\bf Eberly College of Science Research Award} Pennsylvania State University \hfill {November 2016} \\ \\
{\bf Apes Valentes Research Award} Center for Pollinator Research, Penn State \hfill {May 2015} 

\end{rSection}


\end{document}----------------------------

